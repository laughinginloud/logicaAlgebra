\documentclass[10pt]{article}

\pagestyle{empty}

\usepackage{mathtools, amssymb, multicol, proof, xparse}
\usepackage[margin=.2in]{geometry}

\renewcommand{\arraystretch}{1.2}

\newcommand{\inv}[1]{{#1}^{-1}}
\newcommand{\N}{\mathbb{N}}
\newcommand{\oldemptyset}{\emptyset}
\renewcommand{\emptyset}{\varnothing}

\begin{document}

    \section*{Relazioni e funzioni}

        Dati \(n\) insiemi \(A_n\), una relazione \(R\) di arità \(n\) è un insieme nella forma
        \[
            R \coloneqq \left\{(a_1, \, a_2, \, \ldots, \, a_n) \; | \; a_i \in A_i, \, i = 1, \, 2, \, \ldots, \, n\right\}.
        \]
        Dunque, \(R \subseteq A_a \times A_2 \times \ldots \times A_n\).
        Poiché sono insiemi, valgono le usuali operazioni (\(\subset, \, \subseteq, \, =, \, \cap, \, \cup\)).
        Proprietà:

        \begin{table}[h]
            \centering

            \begin{tabular}{|| c | c | c ||}
                \hline
                Proprietà & \(\cap\) & \(\cup\)\\
                \hline\hline
                Idempotenza & \(R \cap R = R\) & \(R \cup R = R\)\\
                \hline
                Commutatività & \(R \cap S = S \cap R\) & \(R \cup S = S \cup R\)\\
                \hline
                Associatività & \((R \cap S) \cap T = R \cap (S \cap T)\) & \((R \cup S) \cup T = R \cup (S \cup T)\)\\
                \hline
                Distributività & \(R \cap (S \cup T) = (R \cap S) \cup (R \cap T)\) & \(R \cup (S \cap T) = (R \cup S) \cap (R \cup T)\)\\
                \hline
            \end{tabular}
        \end{table}

        Il prodotto tra due relazioni \(R \subseteq A_1 \times A_2\) e \(S \subseteq A_2 \times A_3\) è definito come
        \[
            R \cdot S \coloneqq \left\{(a_1, \, a_3) \; | \; \exists \; a_2 \in A_2 \, : (a_1, \, a_2) \in R \land (a_2, \, a_3) \in S\right\}.
        \]
        Il prodotto tra relazioni è associativo, compatibile con l'inclusione (\(R \subseteq T \subseteq A_1 \times A_2 \, \land \,
        S \subseteq V \subseteq A_2 \times A_3 \implies R \cdot S \subseteq T \cdot V \)), ma non è commutativo.

        Relazione inversa: \(\inv{R} \coloneqq \left\{(a_2, \, a_1) \; | \; (a_1, \, a_2) \in R\right\}\).
        Proprietà:

        \medskip

        \begin{minipage}[h]{.5\textwidth}
            \begin{itemize}
                \item \(\inv{(R \cap S)} = \inv{R} \cap \inv{S}\),
                \item \(\inv{(R \cup S)} = \inv{R} \cup \inv{S}\),
                \item \(R \cdot (S \cap T) = (R \cdot S) \cap (R \cdot T)\),
            \end{itemize}
        \end{minipage}
        \begin{minipage}[h]{.5\textwidth}
            \begin{itemize}
                \item \(R \cdot (S \cup T) = (R \cdot S) \cup (R \cdot T)\),
                \item \(\inv{(R \cdot S)} = \inv{S} \cdot \inv{R}\),
                \item \(R \subseteq S \implies \inv{R} \subseteq \inv{S}\).
            \end{itemize}
        \end{minipage}

        \medskip

        La potenza di una relazione binaria \(R \subseteq A \times A\) è definita come
        \[
            R^{n} \coloneqq
            \begin{cases*}
                I_A & \text{se} n = 0\\
                R \cdot R^{n - 1} & \text{se} n > 0
            \end{cases*}
            \qquad \text{dove } n \in \N.
        \]

        La restrizione \(B \subseteq A\) di una relazione \(R \subseteq A \times A\) è definita come
        \(R|_B \coloneqq R \cap (B \times B)\). Proprietà:
        \begin{multicols}{3}
            \begin{itemize}
                \item \(R|_\emptyset = \emptyset\),
                \item \(R|_B \cup R|_C \subseteq R|_{B \cup C}\),
                \item \(R|_B \cap R|_C = R|_{B \cap C}\).
            \end{itemize}
        \end{multicols}

        Proprietà di una relazione binaria \(R \subseteq A \times A\):

        \medskip

        \begin{minipage}[h]{.4\textwidth}
            \begin{description}
                \item[Serialità:] \(\forall \, a \in A, \, \exists \, a' \; | \; (a, \, a') \in R\),
                \item[Riflessività:] \(\forall \, a \in A, \, (a, \, a) \in R\),
                \item[Simmetria:] \(\forall \, (a, \, a') \in R, \, (a', \, a) \in R\),
            \end{description}
        \end{minipage}
        \begin{minipage}[h]{.6\textwidth}
            \begin{description}
                \item[Antisimmetria:] \(\forall \, a, \, a' \in A, \, (a, \, a') \in R \land (a', \, a) \in R \implies a = a'\),
                \item[Transitività:] \(\forall \, (a, \, a'), \, (a', \, a'') \in R, \, (a, \, a'') \in R\).
            \end{description}
        \end{minipage}

        \medskip

        Dato un insieme di proprietà formali \textbf{P} ed una relazione binaria \(R \subseteq A \times A\), si dice \textbf{P}-chiusura
        di \(R\) una relazione \(S \subseteq A \times A\) tale che:
        \begin{itemize}
            \item \(R \subseteq S\),
            \item \(S\) gode di tutte le proprietà in \textbf{P},
            \item se \(T \subseteq A \times A\) gode di tutte le proprietà in \textbf{P} e \(R \subseteq T\) allora \(S \subseteq T\).
        \end{itemize}
        
        Chiusure per \(R \subseteq A \times A\):

        \medskip

        \begin{minipage}[h]{.3\textwidth}
            \begin{description}
                \item[Riflessiva:] \(R \cup I_A\);
                \item[Simmetrica:] \(R \cup \inv{R}\);
                \item[Transitiva:] \(\bigcup_{n > 0} R^n\);
            \end{description} 
        \end{minipage}
        \begin{minipage}[h]{.7\textwidth}
            \begin{description}
                \item[Riflessiva, simmetrica:] \(R \cup I_A \cup \inv{R}\);
                \item[Riflessiva, transitiva:] \(\bigcup_{n \geqslant 0} R^n\);
                \item[Equivalenza (riflessiva, simmetrica, transitiva):] \(\bigcup_{n > 0} {(R \cup I_A \cup \inv{R})}^n\).
            \end{description}
        \end{minipage}

        \medskip

        Data una relazione di equivalenza \(R\), si dice classe di equivalenza
        \([a]_R \coloneqq \left\{x \in A \; | \; (a, \, x) \in R\right\}\).

        Dato un insieme \(A\) ed un insieme non vuoto di indici \(I\), si dice partizione di \(A\) una famiglia
        \(\mathcal{B} \coloneqq \left\{B_i \; | \; i \in I\right\}\) tale che:
        \begin{multicols}{3}
            \begin{itemize}
                \item \(\forall \, i \in I, \, B_i \neq \emptyset\),
                \item \(A = \bigcup_{i \in I} B_i\) (cioè \(\mathcal{B}\) forma un ricoprimento di \(A\)),
                \item \(B_i \cap B_j \neq \emptyset \implies B_i = B_j\).
            \end{itemize}
        \end{multicols}

        Sia \(R\) una relazione di equivalenza su \(A\), la partizione indotta da \(R\) su \(A\) si chiama insieme quoziente
        di \(A\) rispetto a \(R\), dunque \(A/R \coloneqq \left\{[a]_R \; | \; a \in A\right\}\).

        \newpage

        \subsection*{Relazioni d'ordine}

            Una relazione \(R\) su \(A\) che goda delle proprietà riflessiva, antisimmetrica e transitiva si dice relazione
            d'ordine su \(A\). Si indica con \((A, \, \leqslant)\). Se \(\forall \, a, \, b \in A, \, (a, \, b) \in R\), ovvero
            tutti gli elementi sono confrontabili, la relazione d'ordine è totale. In caso contrario viene detta parziale.
            Se \(\forall \, a \in A, \, (a, \, a) \notin R\), ovvero la relazione è antiriflessiva, la relazione si dice d'ordine
            stretto e si indica con \((A, \, <)\).

            \subsubsection*{Minimali e massimali}

                Sia \((A, \, \leqslant)\) un insieme parzialmente ordinato e sia \(m \in A\):
                \begin{itemize}
                    \item se \(\forall \, a \in A, \, m \leqslant a\), \(m\) si dice minimo di \(A\);
                    \item se \(\forall \, a \in A, \, a \leqslant m \implies a = m\), \(m\) si dice minimale di \(A\);
                    \item se \(\forall \, a \in A, \, a \leqslant m\), \(m\) si dice massimo di \(A\);
                    \item se \(\forall \, a \in A, \, m \leqslant a \implies a = m\), \(m\) si dice massimale di \(A\).
                \end{itemize}

                Dato un insieme parzialmente ordinato:
                \begin{itemize}
                    \item se presenta un minimo/massimo, esso è unico;
                    \item se l'insieme è finito e presenta un unico minimale/massimale, esso è il minimo/massimo.
                \end{itemize}

            \subsubsection*{Maggiornati e minoranti}

                Sia \((A, \, \leqslant)\) un insieme parzialmente ordinato, \(B \subseteq A\) e \(m \in A\):
                \begin{itemize}
                    \item se \(\forall \, b \in B, \, m \leqslant b\), \(m\) si dice minorante di \(B\);
                    \item il massimo rispetto a \(\leqslant\) dei minoranti di \(B\) si dice estremo inferiore di \(B\);
                    \item se \(\forall \, b \in B, \, b \leqslant m\), \(m\) si dice maggiornate di \(B\);
                    \item il minimo rispetto a \(\leqslant\) dei maggiornati di \(B\) si dice estremo superiore di \(B\).
                \end{itemize}

            \subsubsection*{Funzioni}

                Data una relazione \(R \subseteq A \times B\), essa si dice:
                \begin{itemize}
                    \item ovunque definita se è seriale rispetto all'insieme \(A\);
                    \item funzionale se per ogni \(a \in A\) esiste al più un \(b \in B\) con cui è in relazione.
                \end{itemize}
                Se una relazione è ovunque definita e funzionale prende il nome di funzione e si indica con \(f : A \to B\). L'insieme
                \(A\) prende il nome di insieme delle immagini e \(B\) di insieme delle controimmagini.

                Una funzione \(f : A \to B\) è detta iniettiva se \(\forall \, a_1, \, a_2 \in A, \, f(a_1) = f(a_2)\implies a_1 = a_2\).
                Si dice suriettiva una funzione tale che \(f(A) = B\). Se una funzione è sia iniettiva che suriettiva si dice biiettiva
                o biunivoca.

                Date due funzioni \(f : A \to B\) e \(g : B \to C\):
                \begin{itemize}
                    \item se \(f\) e \(g\) sono iniettive, allora anche \(f \cdot g\) lo è;
                    \item se \(f \cdot g\) è iniettiva, allora anche \(f\) lo è;
                    \item se \(f\) e \(g\) sono suriettive, allora anche \(f \cdot g\) lo è;
                    \item se \(f \cdot g\) è suriettiva, allora anche \(g\) lo è;
                    \item se \(f\) e \(g\) sono biiettive, allora anche \(f \cdot g\) lo è;
                    \item se \(f \cdot g\) è biiettiva, allora \(f\) è iniettiva e \(g\) è suriettiva.
                \end{itemize}

                Nota: \(f \cdot g = g \circ f\).

                Data una funzione \(f : A \to B\), si dice funzione inversa la funzione \(g : B \to A\) tale che
                \(f \cdot g = I_A\) e \(g \cdot f = I_B\). Nel caso sia ammessa solo un'identità, prende il nome di inversa destra
                o sinistra, rispettivamente.

                Condizione necessaria e sufficiente affinché una funzione ammetta inversa destra è che sia iniettiva. Per quella
                sinistra, è necessario sia suriettiva. Nel caso sia biiettiva, ammette una ed una sola inversa.

                Si dice nucleo o kernel di una funzione la relazione \(\ker_f \subseteq A \times A\) tale che
                \(\forall \, (a_1, \, a_2) \in \ker_f, \, f(a_1) = f(a_2)\).

                Teorema di Cantor: ogni insieme ha cardinalità strettamente inferiore al suo insieme delle parti.

    \section*{Logica proposizionale}

        Alfabeto \(\varepsilon\):
        \begin{itemize}
            \item lettere enunciative, indicate con lettere latine maiuscole;
            \item connettivi logici, composti da \(\neg \,, \, \land, \, \lor, \, \implies, \, \iff\), in questo ordine di precedenza,
                a partire dal più importante;
            \item parentesi tonde.
        \end{itemize}
        Gli operatori associano a sinistra.

        Si dice formula ben formata un formula \(\varphi \Coloneqq A \; | \; \neg \, \varphi \; | \; \varphi \land \varphi \; | \; \varphi \lor \varphi \; | \; \varphi \implies \varphi \; | \; \varphi \iff \varphi\).

        L'insieme \(S(\varphi)\) delle sottoformule di \(\varphi\) è definito come:
        \begin{itemize}
            \item se \(\varphi = A\), \(S(\varphi) = \{\varphi\}\);
            \item se \(\varphi = \neg \, \psi\), \(S(\varphi) = \{\varphi\} \cup S(\psi)\);
            \item se \(\varphi = \psi \circ \vartheta\), \(S(\varphi) = {\varphi} \cup S(\psi) \cup S(\vartheta)\).
        \end{itemize}

        Si dice interpretazione la funzione \(i : \varepsilon \to \{0, \, 1\}\) che agisce sulle singole lettere. Estendendo il concetto
        alle f.b.f. prende il nome di valutazione ed è definita come:
        \begin{itemize}
            \item \(v(A) = i(A)\);
            \item \(v(\neg \, \varphi) = 1 - v(\varphi)\);
            \item \(v(\varphi \land \psi) = \min\{v(\varphi), \, v(\psi)\}\);
            \item \(v(\varphi \lor \psi) = \max\{v(\varphi), \, v(\psi)\}\);
            \item \(v(\varphi \implies \psi) = \max\{1 - v(\varphi), \, v(\psi)\}\);
            \item \(v(\varphi \iff \psi) = 1 - \left|v(\varphi) - v(\psi)\right|\).
        \end{itemize}

        Se \(\varphi\) è una f.b.f. diciamo che una valutazione \(v\) è un suo modello se \(v(\varphi) = 1\). Dunque:
        \begin{itemize}
            \item se esiste almeno un modello per \(\varphi\), questa è detta soddisfacibile;
            \item se ogni valutazione è modello per \(\varphi\), questa è detta tautologia e si indica con \(\models \varphi\);
            \item se nessuna valutazione è modello per \(\varphi\), questa è detta contraddizione od insoddisfacibile.
        \end{itemize}

        Sia \(\Gamma\) un insieme di f.b.f., allora:
        \begin{itemize}
            \item un modello per \(\Gamma\) è modello \(\forall \, \varphi \in \Gamma\);
            \item se \(\Gamma\) ammette un modello è detta soddisfacibile;
            \item se \(\Gamma\) non ammette modelli è detta insoddisfacibile.
        \end{itemize}

        Se ogni modello di \(\Gamma\) è modello di \(\varphi\) questa è detta conseguenza semantica di \(\Gamma\) e si scrive
        \(\Gamma \models \varphi\). In caso contrario si scrive \(\Gamma \not\models \varphi\).

        Date \(\varphi, \, \psi \in \Gamma\), \(\Gamma \cup {\psi} \models \varphi\) sse \(\Gamma \models \psi \implies \varphi\).
        Dunque, \(\psi \models \varphi\) sse \(\models \psi \implies \varphi\) e \(\Gamma \models \varphi\) sse
        \(\Gamma \cup \{\neg \, \varphi\}\) è insoddisfacibile.

        Due f.b.f con modelli equivalenti si dicono semanticamente equivalenti.

        Leggi (ad eccezione delle defizioni di implicazione doppia e singola, è sempre possibile sostituire \(\land\) e \(\lor\)):
        \begin{table}[h]
            \centering

            \begin{tabular}{|| c | c ||}
                \hline
                Formula & Nome\\
                \hline\hline
                \(A \equiv \neg\neg \, A\) & Doppia negazione\\
                \hline
                \(A \land B \equiv B \land A\) & Commutatività\\
                \hline
                \(A \land A \equiv A\) & Idempotenza\\
                \hline
                \((A \land B) \land C \equiv A \land (B \land C)\) & Associatività\\
                \hline
                \(A \land (B \lor C) \equiv (A \land B) \lor (A \land C)\) & Distributività\\
                \hline
                \(A \land (A \lor B) \equiv A\) & Assorbimento\\
                \hline
                \(\neg \, (A \land B) \equiv \neg \, A \lor \neg \, B\) & De Morgan\\
                \hline
                \(A \implies B \equiv \neg \, B \implies \neg \, A\) & Contrapposizione\\
                \hline
                \(A \implies B \equiv \neg \, A \lor B\) & Implicazione\\
                \hline
                \(A \iff B \equiv (A \implies B) \land (B \implies A)\) & Doppia implicazione\\
                \hline
                \((\neg \, A \land A) \lor B \equiv B\) & Non contraddizione\\
                \hline
            \end{tabular}
        \end{table}

        \subsection*{Teoria \(\mathcal{L}\)}

            Tutte le f.b.f. contengono unicamente i connettivi \(\neg\) e \(\implies\).

            Assiomi:
            \begin{enumerate}
                \item \(A \implies (B \implies A)\);
                \item \((A \implies (B \implies C)) \implies ((A \implies B) \implies (A \implies C))\);
                \item \((\neg \, A \implies \neg \, B) \implies ((\neg \, A \implies B) \implies A)\).
            \end{enumerate}

            Modus ponens: da \(A\) e \(A \implies B\) deduco \(B\).

            Teorema di correttezza e completezza: siano \(\varphi, \, \psi \in \Gamma\) due f.b.f., allora
            \begin{description}
                \item[Correttezza] se \(\Gamma \vdash_\mathcal{L} \varphi\), allora \(\Gamma \models \varphi\);
                \item[Completezza] se \(\Gamma \models \varphi\), allora \(\Gamma \vdash_\mathcal{L} \varphi\).
            \end{description}

            Teorema di deduzione sintattica: siano \(\varphi, \, \psi \in \Gamma\) due f.b.f., allora
            \(\Gamma \cup {\psi} \vdash_\mathcal{L} \varphi\) sse \(\Gamma \vdash_\mathcal{L} \psi \implies \varphi\).

        \subsection*{Calcolo per risoluzione}

            Si basa su f.b.f. in forma normale congiuntiva. Ogni disgiunzione finita di letterali è detta clausola, dunque una
            formula è una congiunzione di clausole.

            Regola di risoluzione: date le clausole \(C_1\) e \(C_2\) ed un letterale \(A\), la regola di risoluzione si scrive come
            \[
                \infer{C_1 \cup C_2}{C_1 \cup \{A\} & C_2 \cup \{\neg \, A\}}.
            \]

            Teorema di correttezza e completezza per refutazione:
            \begin{description}
                \item[Correttezza] se \(\Gamma^C \vdash_\mathcal{R} \square\), allora \(\Gamma\) è insoddisfacibile;
                \item[Completezza] se \(\Gamma\) è insoddisfacibile, allora \(\Gamma^C \vdash_\mathcal{R} \square\).
            \end{description}

            L'insieme delle clausole deducibili è definito come \(\mathrm{Ris}\left(\Gamma^C\right) = \Gamma^C \cup \left\{C_{i, \, j} \; | \; C_{i, \, j} \text{ è la risolvente di } C_i \text{ e } C_j\right\}\).

            Algoritmo di risoluzione:
            \begin{enumerate}
                \item \(\Gamma \models \varphi\) sse \(\Gamma \cup \{\neg \, \varphi\}\) è insoddisfacibile;
                \item \(\Gamma\) è insoddisfacibile sse \(\Gamma^C \cup {(\neg \, \varphi)}^C \vdash_\mathcal{R} \square\);
                \item sia \(\Delta = \Gamma^C \cup {(\neg \, \varphi)}^C\), ripetere
                    \begin{enumerate}
                        \item \(\Delta' = \Delta\);
                        \item \(\Delta = \mathrm{Ris}(\Delta)\);
                    \end{enumerate}
                    finché \(\square \notin \Delta\) o \(\Delta \neq \Delta'\);
                \item se \(\square \in \Delta\) allora \(\Gamma \models \varphi\), altrimenti \(\Gamma \not\models \varphi\).
            \end{enumerate}

    \section*{Logica del primo ordine}

        Alfabeto:
        \begin{itemize}
            \item costanti, indicate con le prime lettere latine minuscole;
            \item variabili, indicate con le ultime lettere latine minuscole;
            \item lettere funzionali, indicate con le lettere latine intermedie minuscole;
            \item lettere predicative, indicate con le prime lettere latine maiuscole;
            \item connettivi logici, equivalenti a quelli proposizionali;
            \item quantificatori universale ed esistenziale;
            \item parentesi tonde.
        \end{itemize}

        \newpage

        Si dice segnatura un qualunque insieme \(S\) di costanti, lettere funzionali e predicative. Si dice termine su \(S\) una
        sequenza di costanti, variabili, lettere funzionali e parentesi, dove:
        \begin{itemize}
            \item ogni costante o variabile è un termine;
            \item ogni lettera funzionale con i relativi parametri è un termine.
        \end{itemize}
        Una formula atomica è una formula composta da una lettera predicativa, avente termini come parametri.
        Definizione di f.b.f.: \(\varphi \Coloneqq A \; | \; \neg \, \varphi \; | \; \forall \, x \, \varphi \; | \; \exists \, x \, \varphi \; | \; \varphi \land \varphi \; | \; \varphi \lor \varphi \; | \; \varphi \implies \varphi \; | \; \varphi \iff \varphi\).
        Per l'insieme delle sottoformule, valgono le stesse regole della logica proposizionale, dove
        \(\varphi = Q \, x \, \psi\) diventa \(S(\varphi) = \{\varphi\} \cup S(\psi)\).
        Una variabile si dice legata se è quantificata, si dice libera altrimenti. Se in una formula non esistono variabili libere,
        questa viene detta chiusa; in caso contrario è detta aperta. Una formula chiusa è detta chiusura universale od esistenziale se
        presenta in testa solo quantificatori del rispettivo tipo. Si dice termine libero per una variabile un termine tale che detta
        variabile non sia libera quando un'altra variabile del termine è legata.

        \subsection*{Strutture interpretative}

            Data una segnatura \(S\), si dice struttura interpretativa la coppia \(\left<D; \, I\right>\) dove
            \begin{itemize}
                \item \(D\) è un insieme non vuoto detto dominio;
                \item \(I\) è una funzione di interpretazione, formata da tre funzioni \(I_1\), \(I_2\) e \(I_3\), dove
                    \begin{itemize}
                        \item \(I_1\) associa alle costanti in \(S\) valori in \(D\);
                        \item \(I_2\) associa alle lettere funzionali in \(S\) operazioni in \(D\);
                        \item \(I_3\) associa alle lettere predicative in \(S\) relazioni in \(D\).
                    \end{itemize}
            \end{itemize}

            Si dice assegnamento la funzione \(s : VAR \to D\). Questa può essere estesa ai termini, diventando \(s^* : TER(S) \to D\).

            Una f.b.f. si dice soddisfacibile (in una struttura) se ammette un modello, logicamente valida (vera, in una struttura) se
            ogni interpretazione è modello e logicamente contraddittoria/insoddisfacibile (falsa, in una struttura) se non ammette
            modelli.

            Una formula chiusa può essere solo o vera o falsa, mai soddisfacibile ma non vera. La chiusura universale di una f.b.f. è
            vera sse la formula è vera nella struttura interpretativa. La chiusura esistenziale è vera sse la formula è soddisfacibile.

        \subsection*{Forma di normale prenessa e di Skolem}

            Una formula si dice in forma normale prenessa se presenta tutti i quantificatori in testa alla formula. Regole di
            normalizzazione:
            \begin{enumerate}
                \item \(\neg \, \forall \, x \, \psi \equiv \exists \, x \, \neg \, \psi\);
                \item \(\forall \, x \, \psi \land \vartheta \equiv \forall \, y \left(\psi [y/x] \land \vartheta\right)\);
                \item \(\forall \, x \, \psi \implies \vartheta \equiv \exists \, y \left(\psi [y/x] \implies \vartheta\right)\);
                \item \(\vartheta \implies \forall \, x \, \psi \equiv \forall \, y \left(\vartheta \implies \psi [y/x]\right)\).
            \end{enumerate}
            È sempre possibile invertire \(\forall/\exists\) e \(\land/\lor\) in queste regole.
            
            Una formula si dice
            in forma di Skolem se è prenessa e presenta solo quantificatori universali. Skolemizzazione:
            \begin{enumerate}
                \item si porta la formula in forma prenessa;
                \item si elimina ogni esistenziale in testa, sostituendo le varibili con delle costanti;
                \item si elimina ogni esistenziale successivo agli universali, sostituendo le varibili con lettere funzionali di
                    arità pari al numero di universali che precedono, con le relative variabili come parametri.
            \end{enumerate}

            Una f.b.f. è insoddisfacibile sse una forma di Skolem della sua chiusura universale è insoddisfacibile.

            Data una formula in forma di Skolem, è possibile risolverla mediante metodi equivalenti a quelli della logica
            proposizionale, applicando opportune sostituzioni di variabili. 

        \subsection*{Teoria \(\mathcal{K}\)}

            Equivalente alla teoria \(\mathcal{L}\), con l'aggiunta di \(\forall\).

            Assiomi aggiuntivi:
            \begin{enumerate}
                \item \(\forall \, x \, A \implies A \, [t/x]\);
                \item \(\forall \, x \left(A \implies B\right) \implies \left(A \implies \forall \, x \, B\right)\).
            \end{enumerate}
            Oltre al modus ponens, esiste la generalizzazione: \(A\) diventa \(\forall \, x \, A\).

    \section*{Strutture algebriche}

        \begin{description}
            
            \item[Semigruppo] Una struttura algebrica nella forma \((A, +)\), dove \(+\) è un'operazione associativa sull'insieme \(A\);
            \item[Semigruppo commutativo] Un semigruppo con l'aggiunta della proprietà commutativa;
            \item[Monoide] Un semigruppo con l'aggiunta di un'elemento neutro (scrittura: \((A, +, u)\));
            \item[Gruppo] Un monoide che presenta l'inverso per ogni elemento dell'insieme;
            \item[Gruppo abeliano] Un gruppo che presenta la proprietà commutativa;
            \item[Anello] Una struttura nella forma \((A, +, *, u)\), dove
                \begin{itemize}
                    \item \((A, +, u)\) forma un gruppo abeliano;
                    \item \((A, *)\) forma un semigruppo;
                    \item \(*\) è distributiva rispetto a \(+\);
                \end{itemize}
            \item[Anello con unità] Un anello con un monoide al posto del semigruppo;
            \item[Anello commutativo] Un anello dove il semigruppo è commutativo;
            \item[Corpo] Un anello dove \((A \backslash \{u\}, *)\) forma un gruppo;
            \item[Campo] Un corpo dove \((A \backslash \{u\}, *)\) forma un gruppo abeliano.

        \end{description}

        Note:
        \begin{itemize}
            \item in un gruppo, l'inverso di un elemento è sempre unico;
            \item in un gruppo, vale la legge di cancellazione, ovvero \(a + b = b + c \equiv b = c\);
            \item in un anello, \(a * u = u * a = u\);
            \item in un anello, \(a * b = a * c \equiv b = c\) dove \(a \neq u\);
            \item in un anello, si dice divisore dello zero un elemento \(a\) tale che esiste \(b\) per
                cui \(ab = 0\), dove \(a \neq 0\), \(b \neq 0\);
            \item teorema di Lagrange: dato un gruppo di ordine \(n\), ogni suo sottogruppo ha ordine che divide \(n\).
        \end{itemize}

        \subsection*{Sottostrutture e morfismi}

            Criteri sottostrutture:
            \begin{itemize}
                %\item per i sottogruppi, controllare se \(a + b\) e \(-a\) appartengono;
                \item per i sottogruppi, controllare se \(a - b\) appartiene (oppure \(a \cdot b^{-1}\));
                \item per i sottoanelli, controllare se \(a - b\) e \(a * b\) appartengono.
            \end{itemize}

            Per controllare se è presente un omomorfismo, dati \((A, \, +, \, u)\), \((\hat{A}, \, \hat{+}, \, \hat{u})\) e
            \(f : A \to \hat{A}\), controllare se \(f(a + b) = f(a) \hat{+} f(b)\) e \(f(u) = \hat{u}\).

    \section*{SPASS}

            \begin{minipage}[h]{.4\textwidth}
                Esempio:
                \begin{verbatim}
list_of_symbols.
    functions[(f, 0), (g, 1)].
    predicates[(A, 0), (B, 1)].
end_of_list.

list_of_formulae(axioms).
    formula(...).
end_of_list.

list_of_formulae(conjectures).
    formula(...).
end_of_list.
                \end{verbatim}
            \end{minipage}
            \begin{minipage}[h]{.6\textwidth}
                Funzioni standard:
                \begin{description}
                    \item[\texttt{lnot(fbf)}] not logico;
                    \item[\texttt{and(fbf, fbf [, fbf...])}] and logico;
                    \item[\texttt{or(fbf, fbf [, fbf...])}] or logico;
                    \item[\texttt{implies(fbf, fbf)}] implicazione;
                    \item[\texttt{equiv(fbf, fbf)}] doppia implicazione;
                    \item[\texttt{equal(fbf, fbf)}] uguaglianza;
                    \item[\texttt{forall([var [, var...]], fbf)}] per ogni;
                    \item[\texttt{exists([var [, var...]], fbf)}] esiste;.
                \end{description}
            \end{minipage}

\end{document}